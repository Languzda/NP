\subsubsection{Prosta formuła}

\begin{equation}
	A = \frac{t_H - t_L}{t_H+t_L} \cdot 100
\label{eq:asym}
\end{equation}

\subsubsection{Złożona formuła}
\begin{equation} 
S(\omega)=1.466\, H_s^2 \,  \frac{\omega_0^5}{\omega^6}  \, e^{\left[-3^{\omega/(\omega_0)}\right]^2}
\label{eq:S}
\end{equation}

\begin{equation} 
\lim_{x\to 0}{\frac{e^x-1}{2x}}
 \overset{\left[\frac{0}{0}\right]}{\underset{\mathrm{H}}{=}}
 \lim_{x\to 0}{\frac{e^x}{2}}={\frac{1}{2}}
\label{eq:lim}
\end{equation}

\subsubsection{Opisy w formule}

\begin{equation} 
z = \overbrace{
   \underbrace{x}_\text{real} + i
   \underbrace{y}_\text{imaginary}
  }^\text{complex number}
	\label{eq:CN}
\end{equation}

\subsubsection{Układy równań}

Na podstawie: \cite{noauthor_latexadvanced_nodate},\cite{noauthor_numbering_nodate}.

\begin{equation} 
\begin{aligned}[b]
 f(x)  &= a x^2+b x +c   &   g(x)  &= d x^3 \\
 f'(x) &= 2 a x +b       &   g'(x) &= 3 d x^2
\end{aligned}
	\label{eq:aligned}
\end{equation}

Zaczerpnięto z \cite{noauthor_latex_nodate}.

\begin{equation} 
y = \left\{ \begin{array}{ll}
a & \textrm{gdy $d>c$}\\
b+x & \textrm{gdy $d=c$}\\
l & \textrm{gdy $ d < c $}
\end{array} \right.
	\label{eq:eqnarray}
\end{equation}


\subsubsection{Macierze}

Przed słowem kluczowym \textit{matrix} należy umieścić literę odpowiadającą rodzajowi wymaganych znaczników \cite{noauthor_how_nodate}:
\begin{itemize}
	\item \textbf{p} -- nawiasy okrągłe
	\item \textbf{b} -- nawiasy kwadratowe
	\item \textbf{v} -- pionowe linie (wyznacznik)
	\item \textbf{B} -- klamry
	\item \textbf{V} -- podwójne pionowe linie
\end{itemize}


Przykład (\textit{matrix}):

\begin{equation}
A_{m,n} = 
\begin{matrix}
a_{1,1} & a_{1,2} & \cdots & a_{1,n} \\
a_{2,1} & a_{2,2} & \cdots & a_{2,n} \\
\vdots  & \vdots  & \ddots & \vdots  \\
a_{m,1} & a_{m,2} & \cdots & a_{m,n} 
\end{matrix}
\end{equation}


Przykład (\textit{pmatrix}):

\begin{equation}
A_{m,n} = 
\begin{pmatrix}
a_{1,1} & a_{1,2} & \cdots & a_{1,n} \\
a_{2,1} & a_{2,2} & \cdots & a_{2,n} \\
\vdots  & \vdots  & \ddots & \vdots  \\
a_{m,1} & a_{m,2} & \cdots & a_{m,n} 
\end{pmatrix}
\end{equation}



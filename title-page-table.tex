\setlength\extrarowheight{2pt}
\begin{table}[ht]
\centering
%\resizebox{\textwidth}{!}{%
\begin{tabular}{|p{4cm}|p{8cm}|p{2cm}|}
\hline
%\rowcolor{gray}
%\multicolumn{3}{|c|}{\textcolor[rgb]{1,1,1}{Scalenie trzech kolumn}}\\

\multicolumn{2}{|c|}{\cellcolor{gray-pp}\textcolor[rgb]{1,1,1}{Politechnika Poznańska}}  & \multicolumn{1}{c|}{\multirow{3}{*}{\resizebox{15mm}{!}{\includegraphics{img/logo2.eps}}}}\\ 
\multicolumn{2}{|c|}{\cellcolor{gray-pp}\textcolor[rgb]{1,1,1}{Wydział Automatyki, Robotyki i Elektrotechniki}} & \\ 
\multicolumn{2}{|c|}{\cellcolor{gray-pp}\textcolor[rgb]{1,1,1}{Instytut Robotyki i Inteligencji Maszynowej}} & \\ 
\hline 
\multicolumn{1}{|c|}{Dz>AiR>Sem3} & \multicolumn{1}{c|}{Napędy przekształtnikowe} & \multicolumn{1}{c|}{2020/21 (s.zim.)} \\
\hline
Skład osobowy: \par Robert Dobrowolski \par Ksawery Giera \par Krzysztof Nosal & \textbf{Model Obwodowy silnika prądu stałego} & Data wyk.:\par 25.10.22\\
\hline
Grupa A5/L9  & Ćwiczenie 2 & Zajęcia 3 \\
\hline
\end{tabular}%
%}
\end{table}	
\setlength\extrarowheight{0pt}
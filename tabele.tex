Materiały źródłowe dot.formatowania tabel:\cite{noauthor_multi-column_nodate}, \cite{noauthor_verbatim_nodate} oraz:

\begin{itemize}
	\item \url{http://ftp.icm.edu.pl/packages/GUST/bulletin/19/sapij03b.pdf}
	\item \url{https://sunsite.icm.edu.pl/pub/CTAN/macros/latex/contrib/colortbl/colortbl.pdf}
	\item \url{https://texblog.org/2017/12/12/color-table-series-part-1-introduction-colortbl-package/}
\end{itemize}


\begin{table}[ht]
\centering
\caption{Przykładowy opis tabeli}
\label{tab:example}
%\resizebox{\textwidth}{!}{%
\begin{tabular}{|c|c|p{5cm}|}
\hline
\rowcolor{gray}
\multicolumn{3}{|c|}{\textcolor[rgb]{1,1,1}{Scalenie trzech kolumn}}\\
\hline
NL27WZ16DFT2G   & FODM8071   & większa, sterowana szerokość kolumny  \\ 
\hline
\end{tabular}%
%}
\end{table}	

Do tabel jak tej powyższej \ref{tab:example} odwołujemy się przez słowo kluczowe \textit{ref} z numerem pola w opcji \textit{label}.

W sekcji wskazano dwa przykłady listingu, przy czym kody można raportować dwojako \cite{carsten_heinz_listings_nodate} -- jawnie w źródle dokumentu \LaTeX\ (listing \ref{lst:C01}) oraz pośrednio poprzez podanie ścieżki do pliku z kodem (listing \ref{lst:Mat01}). 

\subsubsection{Język C}

\begin{lstlisting}[language=C, caption = {Przykładowy listing języka \textit{C}}, label = {lst:C01}]

/**
  ******************************************************************************
	* @file           : main.c
	* @brief          : Main program body
	* @author 				: Bogdan Fabianski
  ******************************************************************************
**/

/* Includes ------------------------------------------------------------------*/
#include "main.h"
#include "stm32h7xx_hal.h"
#include "cmsis_os.h"
#include "stm32h753xx.h"
#include "SEGGER_RTT.h"

/* Private variables ---------------------------------------------------------*/
char log_string[512];
extern uint32_t __initial_sp;
extern __IO uint8_t is_PBA_on;

/**
  * @brief System main entrance -- configuration and start to run
  * @retval None
  *
**/

int main(void)
{
#ifndef NO_CACHE
/* MPU Configuration----------------------------------------------------------*/
MPU_Config();

  /* Enable I-Cache---------------------------------------------------------*/
  SCB_EnableICache();

  /* Enable D-Cache---------------------------------------------------------*/
  SCB_EnableDCache();
#endif
...

  /* Start scheduler */
  osKernelStart();

  /* Infinite loop */
  /* USER CODE BEGIN WHILE */
  while(1){}
}

\end{lstlisting}

\subsubsection{Język skryptowy Matlab}

\lstinputlisting[language=Matlab, frame=single, caption = {Przykładowy listing języka skryptowego \textit{Matlab}}, label = {lst:Mat01}]{src/matlab-example.m}



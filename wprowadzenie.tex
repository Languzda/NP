\subsection{Cel opracowania}

Celem zajęć jest przybliżenie metodyki postępowania w przypadku syntezy prostego modelu silnika prądu
stałego (ang. DCM – Direct Current Motor). Model ten będzie stanowił podstawę dalszych rozważań
stąd uznano, iż omówienie tego tematu dla zrozumienia kolejnych ćwiczeń będzie niezbędne.

Na wstępie należy przyjąć, iż każdy model stanowi jedynie (mniej lub bardziej dokładne/zbieżne)
przybliżenie rzeczywistości. Dokładność modelu zależy od jakości identyfikacji struktury/złożoności
obiektu (przyjętych równań matematycznych, najczęściej w postaci równań różniczkowych) oraz
identyfikacji wartości jego parametrów. W niniejszym ćwiczeniu zostanie przedstawione zagadnienie
syntezy struktury silnika bazując na analizie zjawisk fizycznych. Kwestia identyfikacji wartości
parametrów będzie omówiona w kolejnej instrukcji.

\subsection{Narzędzia i źródła}

Symulacje będą przedstawiane w środowiskach:

\begin{itemize}
	\item komercyjnym: Matlab/Simulink (licencja trial, sieciowa w ramach zajęć stacjonarnych),
	\item lub/oraz bezpłatnym, (typu open-source): SciLab [1][2](\url{https://www.scilab.org/})/Xcos.
\end{itemize}

Możliwe jest wykorzystanie innych platform/narzędzi, a przykładowe ich zestawienie można znaleźć
na stronie \url{https://www.g2.com/products/simulink/competitors/alternatives}.